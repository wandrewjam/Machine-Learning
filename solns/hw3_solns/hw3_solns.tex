%%%%%%%%%%%%%%%%%%%%%%%%%%%%%%%%%%%%%%%%%
% Structured General Purpose Assignment
% LaTeX Template
%
% This template has been downloaded from:
% http://www.latextemplates.com
%
% Original author:
% Ted Pavlic (http://www.tedpavlic.com)
%
% Note:
% The \lipsum[#] commands throughout this template generate dummy text
% to fill the template out. These commands should all be removed when 
% writing assignment content.
%
%%%%%%%%%%%%%%%%%%%%%%%%%%%%%%%%%%%%%%%%%

%----------------------------------------------------------------------------------------
%	PACKAGES AND OTHER DOCUMENT CONFIGURATIONS
%----------------------------------------------------------------------------------------

\documentclass{article}

\usepackage{fancyhdr} % Required for custom headers
\usepackage{lastpage} % Required to determine the last page for the footer
\usepackage{extramarks} % Required for headers and footers
\usepackage{graphicx} % Required to insert images
\usepackage{lipsum} % Used for inserting dummy 'Lorem ipsum' text into the template
\usepackage{amsmath, amssymb, amsthm}
\usepackage{tikz}

%%%%%%%% Convenient Commands %%%%%%%
\newcommand{\Z}{\mathbb{Z}}
\newcommand{\R}{\mathbb{R}}
\newcommand{\C}{\mathbb{C}}
\newcommand{\der}[2]{\frac{d #1}{d #2}}
\newcommand{\pder}[2]{\frac{d #1}{d #2}}
\newcommand{\inv}{^{-1}}
\newcommand{\mat}[1]{\textbf{#1}}
\newcommand{\eps}{\varepsilon}
\newcommand{\ds}{\displaystyle}
\newcommand{\abs}[1]{\lvert #1 \rvert}
\newcommand{\tn}{\textnormal}
\DeclareMathOperator{\sign}{sign}
\renewcommand{\labelenumi}{[\textbf{\alph{enumi}}]}

% %%%%%%%%% Theorem Commands %%%%%%%%%
% \newtheorem{thm}{Theorem}[section] %this creates theorems
% \newtheorem{cor}[thm]{Corollary}
% \newtheorem{prop}[thm]{Proposition}
% \newtheorem{dfn}[thm]{Definition}
% \newtheorem{lem}[thm]{Lemma}
% \newtheorem{rmk}[thm]{Remark}
% \newtheorem{exm}{Example}[section]

% Margins
\topmargin=-0.45in
\evensidemargin=0in
\oddsidemargin=0in
\textwidth=6.5in
\textheight=9.0in
\headsep=0.25in 

\linespread{1.1} % Line spacing

% Set up the header and footer
\pagestyle{fancy}
\lhead{\hmwkAuthorName} % Top left header
\chead{\hmwkClass\ (\hmwkClassInstructor\ \hmwkClassTime): \hmwkTitle} % Top center header
\rhead{\firstxmark} % Top right header
\lfoot{\lastxmark} % Bottom left footer
\cfoot{} % Bottom center footer
\rfoot{Page\ \thepage\ of\ \pageref{LastPage}} % Bottom right footer
\renewcommand\headrulewidth{0.4pt} % Size of the header rule
\renewcommand\footrulewidth{0.4pt} % Size of the footer rule

\setlength\parindent{0pt} % Removes all indentation from paragraphs

%----------------------------------------------------------------------------------------
%	DOCUMENT STRUCTURE COMMANDS
%	Skip this unless you know what you're doing
%----------------------------------------------------------------------------------------

% Header and footer for when a page split occurs within a problem environment
\newcommand{\enterProblemHeader}[1]{
\nobreak\extramarks{#1}{#1 continued on next page\ldots}\nobreak
\nobreak\extramarks{#1 (continued)}{#1 continued on next page\ldots}\nobreak
}

% Header and footer for when a page split occurs between problem environments
\newcommand{\exitProblemHeader}[1]{
\nobreak\extramarks{#1 (continued)}{#1 continued on next page\ldots}\nobreak
\nobreak\extramarks{#1}{}\nobreak
}

\setcounter{secnumdepth}{0} % Removes default section numbers
\newcounter{homeworkProblemCounter} % Creates a counter to keep track of the number of problems

\newcommand{\homeworkProblemName}{}
\newenvironment{homeworkProblem}[1][Problem \arabic{homeworkProblemCounter}]{ % Makes a new environment called homeworkProblem which takes 1 argument (custom name) but the default is "Problem #"
\stepcounter{homeworkProblemCounter} % Increase counter for number of problems
\renewcommand{\homeworkProblemName}{#1} % Assign \homeworkProblemName the name of the problem
\subsection{\homeworkProblemName} % Make a section in the document with the custom problem count
\enterProblemHeader{\homeworkProblemName} % Header and footer within the environment
}{
\exitProblemHeader{\homeworkProblemName} % Header and footer after the environment
}

\newcommand{\problemAnswer}[1]{ % Defines the problem answer command with the content as the only argument
\noindent\framebox[\columnwidth][c]{\begin{minipage}{0.98\columnwidth}#1\end{minipage}} % Makes the box around the problem answer and puts the content inside
}

\newcommand{\homeworkSectionName}{}
\newenvironment{homeworkSection}[1]{ % New environment for sections within homework problems, takes 1 argument - the name of the section
\renewcommand{\homeworkSectionName}{#1} % Assign \homeworkSectionName to the name of the section from the environment argument
\subsubsection{\homeworkSectionName} % Make a subsection with the custom name of the subsection
\enterProblemHeader{\homeworkProblemName\ [\homeworkSectionName]} % Header and footer within the environment
}{
\enterProblemHeader{\homeworkProblemName} % Header and footer after the environment
}
   
%----------------------------------------------------------------------------------------
%	NAME AND CLASS SECTION
%----------------------------------------------------------------------------------------

\newcommand{\hmwkTitle}{Homework\ \#3} % Assignment title
\newcommand{\hmwkDueDate}{Monday,\ October\ 16,\ 2017} % Due date
\newcommand{\hmwkClass}{CaltechX: Learning From Data} % Course/class
\newcommand{\hmwkClassTime}{} % Class/lecture time
\newcommand{\hmwkClassInstructor}{Yaser Abu-Mostafa} % Teacher/lecturer
\newcommand{\hmwkAuthorName}{Andrew Watson} % Your name

%----------------------------------------------------------------------------------------
%	TITLE PAGE
%----------------------------------------------------------------------------------------

\title{
\vspace{2in}
\textmd{\textbf{\hmwkClass:\ \hmwkTitle}}\\
\normalsize\vspace{0.1in}\small{Due\ on\ \hmwkDueDate}\\
\vspace{0.1in}\large{\textit{\hmwkClassInstructor\ \hmwkClassTime}}
\vspace{3in}
}

\author{\textbf{\hmwkAuthorName}}
\date{} % Insert date here if you want it to appear below your name

%----------------------------------------------------------------------------------------

\begin{document}

\maketitle

%----------------------------------------------------------------------------------------
%	TABLE OF CONTENTS
%----------------------------------------------------------------------------------------

%\setcounter{tocdepth}{1} % Uncomment this line if you don't want subsections listed in the ToC

\newpage
\tableofcontents
\newpage


\section{Generalization Error}

%----------------------------------------------------------------------------------------
%	PROBLEM 1
%----------------------------------------------------------------------------------------

\begin{homeworkProblem}
The modified Hoeffding Inequality provides a way to characterize the generalization error with a probabilistic bound
	$$\mathbb{P}[|E_\tn{in}(g) - E_\tn{out}(g)| > \epsilon] \le 2M e^{-2\epsilon^2 N}$$
for any $\epsilon > 0$. If we set $\epsilon = 0.05$ and want the probability bound $2M e^{2\epsilon^2 N}$ to be at most 0.03, what is the least number of examples $N$ (among the given choices) needed for the case $M = 1$?

\begin{enumerate}
	\item 500
	\item 1000
	\item 1500
	\item 2000
	\item More examples are needed
\end{enumerate} % Question

\problemAnswer{ % Answer
	To solve this, we can just plug in the values we are given, and then solve for $N$:
	\begin{align*}
		\mathbb{P}[|E_\tn{in}(g) - E_\tn{out}(g)| > \epsilon] &\le 2M e^{-2\epsilon^2 N} \\
		&= 2e^{-2(0.05)^2 N} = 2e^{-0.005 N} \le 0.03 \\
		&\implies N \ge 840.
	\end{align*}
	
	Therefore [\textbf{b}] is the correct answer.
}
\end{homeworkProblem}

%----------------------------------------------------------------------------------------
%	PROBLEM 2
%----------------------------------------------------------------------------------------

\begin{homeworkProblem}
Repeat for the case $M = 10$.

\begin{enumerate}
	\item 500
	\item 1000
	\item 1500
	\item 2000
	\item More examples are needed
\end{enumerate} % Question

\problemAnswer{ % Answer
	For any value of $M$ (but still with $\epsilon = 0.05$), we require that $N \ge -\frac{\ln{(0.015/M)}}{0.005}$. By plugging $M = 10$ into this formula, we see that now the requirement is $N \ge 1301$, and therefore the answer is [\textbf{c}].
}

\end{homeworkProblem}

%----------------------------------------------------------------------------------------
%	PROBLEM 3
%----------------------------------------------------------------------------------------

\begin{homeworkProblem}
Repeat for the case $M = 100$.

\begin{enumerate}
	\item 500
	\item 1000
	\item 1500
	\item 2000
	\item More examples are needed
\end{enumerate} % Question

\problemAnswer{ % Answer
	Finally, plugging $M = 100$ into the formula above, we find that $N \ge 1761$. Therefore the answer is [\textbf{d}].
}
\end{homeworkProblem}

\section{Break Point}

%----------------------------------------------------------------------------------------
%	PROBLEM 4
%----------------------------------------------------------------------------------------

\begin{homeworkProblem}
As shown in class, the (smallest) break point for the Perceptron Model in the two-dimensional case ($\R^2$) is 4 points. What is the smallest break point for the Perceptron Model in $\R^3$? (i.e., instead of the hypothesis set consisting of separating lines, it consists of separating planes.)

\begin{enumerate}
	\item 4
	\item 5
	\item 6
	\item 7
	\item 8
\end{enumerate} % Question

\problemAnswer{ % Answer
	
}
\end{homeworkProblem}

\section{Growth Function}

%----------------------------------------------------------------------------------------
%	PROBLEM 5
%----------------------------------------------------------------------------------------

\begin{homeworkProblem}
Which of the following are possible formulas for a growth function $m_\mathcal{H}(N)$:
\begin{enumerate}
	\item[i)] $1 + N$
	\item[ii)] $1 + N + {N}\choose{2}$
	\item[iii)] $\sum_{i=1}^{\lfloor \sqrt{N} \rfloor} {N}\choose{i}$
	\item[iv)] $2^{\lfloor N/2 \rfloor}$
	\item[v)] $2^N$
\end{enumerate}
where $\lfloor u \rfloor$ is the biggest integer $\leq u$, and ${M}\choose{m} = 0$ when $m > M$.

\begin{enumerate}
	\item i, v
	\item i, ii, v
	\item i, iv, v
	\item i, ii, iii, v
	\item i, ii, iii, iv, v
\end{enumerate} % Question

\problemAnswer{ % Answer
	
}
\end{homeworkProblem}

\section{Fun with Intervals}

%----------------------------------------------------------------------------------------
%	PROBLEM 6
%----------------------------------------------------------------------------------------

\begin{homeworkProblem}
Consider the ``2-intervals'' learning model, where $h: \R \rightarrow \{-1, +1\}$ and $h(x) = +1$ if the point is within either of two arbitrarily chosen intervals and $-1$ otherwise. What is the (smallest) break point for this hypothesis set?

\begin{enumerate}
	\item 3
	\item 4
	\item 5
	\item 6
	\item 7
\end{enumerate} % Question

\problemAnswer{ % Answer
	
}
\end{homeworkProblem}

%----------------------------------------------------------------------------------------
%	PROBLEM 7
%----------------------------------------------------------------------------------------

\begin{homeworkProblem}
Which of the following is the growth function $m_\mathcal{H}(N)$ for the ``2-intervals'' hypothesis set?

\begin{enumerate}
	\item ${N+1}\choose{4}$
	\item ${N+1}\choose{2} + 1$
	\item ${{N+1}\choose{4}} + {{N+1}\choose{2}} + 1$
	\item ${{N+1}\choose{4}} + {{N+1}\choose{3}} + {{N+1}\choose{2}} + {{N+1}\choose{1}} + 1$
	\item None of the above
\end{enumerate} % Question

\problemAnswer{ % Answer
	
}
\end{homeworkProblem}

%----------------------------------------------------------------------------------------
%	PROBLEM 8
%----------------------------------------------------------------------------------------

\begin{homeworkProblem}
Now, consider the general case: the ``M-intervals'' learning model. Again $h: \R \rightarrow \{-1, +1\}$, where $h(x) = +1$ if the point falls inside any of $M$ arbitrarily chosen intervals, otherwise $h(x) = -1$. What is the (smallest) break point of this hypothesis set?

\begin{enumerate}
	\item $M$
	\item $M + 1$
	\item $M^2$
	\item $2M + 1$
	\item $2M - 1$
\end{enumerate} % Question

\problemAnswer{ % Answer
	
}
\end{homeworkProblem}

\section{Convex Sets: The Triangle}

%----------------------------------------------------------------------------------------
%	PROBLEM 9
%----------------------------------------------------------------------------------------

\begin{homeworkProblem}
Consider the ``triangle'' learning model, where $h: \R^2 \rightarrow \{-1, +1\}$ and $h(\mathbf{x}) = +1$ if $\mathbf{x}$ lies within an arbitrarily chosen triangle in the plane and -1 otherwise. Which is the largest number of points in $\R^2$ (among the given choices) that can be shattered by this hypothesis set?

\begin{enumerate}
	\item 1
	\item 3
	\item 5
	\item 7
	\item 9
\end{enumerate} % Question

\problemAnswer{ % Answer
	
}
\end{homeworkProblem}

\section{Non-Convex Sets: Concentric Circles}

%----------------------------------------------------------------------------------------
%	PROBLEM 10
%----------------------------------------------------------------------------------------

\begin{homeworkProblem}
Compute the growth function $m_\mathcal{H}(N)$ for the learning model made up of two concentric circles around the origin in $\R^2$. Specifically, $\mathcal{H}$ contains the functions which are +1 for
	$$a^2 \leq x_1^2 + x_2^2 \leq b^2$$
and -1 otherwise, where $a$ and $b$ are the model parameters. The growth function is

\begin{enumerate}
	\item $N + 1$
	\item ${N+1}\choose{2} + 1$
	\item ${N+1}\choose{3} + 1$
	\item $2N^2 + 1$
	\item None of the above
\end{enumerate} % Question

\problemAnswer{ % Answer
	
}
\end{homeworkProblem}

%----------------------------------------------------------------------------------------

\end{document}
